\end{center}\end{titlepage}\setcounter{tocdepth}{1}\tableofcontents
\chapter{Introduction}
Je m'appelle Romain Brouard. Je suis actuellement en formation ISIE (Informatique et Systèmes Intelligents Embarqués) à Polytech Tours.
J'ai choisi cette formation car je considère qu'il est important pour un informaticien de comprendre comment son outil de travail fonctione. Dès lors, une formation alliant à la fois informatique et électronique m'a semblé être le meilleur compromis.

Eiffage est un groupe français de BTP et concessions. Pour diversifier ses activitées, le groupe est subdivisé en plusieurs branches, parmis lesquelles celle spécialisée dans l'énergie : Eiffage Energie Systèmes (EES). Les principales activités d'EES sont le génie électriquie, le génie climatique, le génie mécanique et les télécommunications. La Direction Régionale Centre Normandie regroupe 18 agences en région Centre et Normandie. J'effectue mon alternance dans l'agence Centre Loire située à Orléans dans le Loiret. Les principales activités de cette agence sont le génie électrique, la maintenance, les réseaux et systèmes d'informations, le génie mécanique, le génie climatique, la métallerie et la couverture. Je passe la plupart de mon temps au bureau, puisque j'appartiens au d'étude Automatisme-Electricité-Informatique de la section Automatisme, Eléctricité, Informatique et Métallerie. Il m'arrive néanmoins d'aller sur chantiers, notamment lors de la configuration et mise en service des bornes des zones d'accès piétons, ou pour des dépanages tel que celui au synchrotron soleil (voir semaine 45).